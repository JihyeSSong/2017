\section{A Large Ion Collider Experiment at the LHC}
ALICE (A Large Ion Collider Experiment) has been collecting data during the whole first phase of the Large Hadron Collider operations, from its startup on the 23 November 2009 to the beginning of the first long technical shutdown in February 2013. During the first three years of operations LHC provided pp collisions at 0.9, 2.76, 7 and 8 TeV, Pb?Pb collisions at 2.76A TeV and finally p?Pb collisions at 5.02 TeV. The first section of this chapter focuses on the LHC performance during this phase and includes details on the accelerator parameters that allow the LHC to perform as a lead ion collider. A detailed description of the ALICE detector fol- lows in the section 2.2. ALICE has been designed and optimized to study the high particle-multiplicity environment of ultra-relativistic heavy-ion collisions and its tracking and particle identification performance in Pb-Pb collisions are discussed. The attention is drawn in particular on the central barrel detectors. Section 2.3 de- scribes the ALICE Data Acquisition (DAQ) system, that also embeds tools for the online Data Quality Monitoring (DQM). The final part of the chapter is dedicated to the offline computing and reconstruction system based on the GRID framework.


\subsection{The Large Hadron Collider}
The Large Hadron Collider (LHC) [57], [58] is a two-ring-superconducting hadron accelerator and collider installed in the 26.7 Km tunnel that hosted the LEP ma- chine and it completes the CERN accelerator complex together with the PS and SPS, among the others shown in fig. 2.1. Four main experiments are located in four different interaction points along its circumference. ATLAS and CMS, the biggest ones, are multi-purpose detectors built to discover the Higgs boson and hints of new physics beyond the Standard Model. LHCb is dedicated to the physics of the flavour, focusing on the study CP-violation using B meson decay channels. The phenomena that these three experiments aim to observe have production cross sec- tion of the order of a hundred of pb or lower, therefore a large number of collision events is required to the machine in order to fulfill the LHC pp physics program. ALICE, on the contrary, is dedicated to the physics of Quark Gluon Plasma through the observation of high-energy heavy-ion collisions, although a shorter physics pro- gram with pp collisions has been carried out.

\subsection{The ALICE project}
\subsubsection{ALICE detector}
\subsubsection{Data Acquisition (DAQ) and trigger system}
\subsubsection{ALICE offline software frame work}
