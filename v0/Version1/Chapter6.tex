\section{Measurement of \xis production in Pb--Pb}
The measurement of resonance production in ultra-relativistic heavy-ion collisions provides information on the properties of the hadronic medium and different stages of its evolution. The measurement of the short-lived resonances allows to estimate the time span in the hadronic phase between the chemical and the kinetic freeze- out. 
In order to study the particle production mechanism, the \xis resonance production at mid-rapidity is measured in p--Pb collisions \snn = 5.02  TeV and in Pb-Pb collisions at \snn = 2.76 TeV with the ALICE experiment, via the reconstruction of its hadronic decay into $\Xi\pi$.

\subsection{\xis-reconstruction}
The \xiss production in p--Pb collisionsat \snn = 5.02 TeV and Pb--Pb interactions at \snn = 2.76 TeV at mid-rapidity has been studied in different multiplicity(p--Pb) and centrality(Pb--Pb) classes, from very central to peripheral collisions. The analysis strategy is based on the invariant mass study of the reconstructed pairs (referred to as the candidates) whose provenance could be the decay of a \xiss baryon into charged particles. The decay products (also called daughters in the text) are identified as oppositely charged $\Xi$ and $\pi$ among the tracks reconstructed in the central barrel. The event selection, track selection and the particle identification strategy is described. The raw signal yield is extracted by fitting the background-subtracted invariant mass distribution in several transverse momentum intervals. In order to extract the \pt-dependent cross section, these yields are corrected for efficiency. The \pt-dependent correction due to the detector acceptance and reconstruction efficiency, (Acc $\times$ $\epsilon_{rec}$)(pt), is computed from a Monte Carlo simulation. The absolute normalisation is then performed, by dividing for the number of the events in each multiplicity/centrality classes.


\subsubsection{Data sample and event selection}
The data used in this note come from p--Pb, \snn = 5.02 TeV collisions measured by the ALICE experiment during the 2013 run at the LHC. About 111.1$\times$10$^{6}$ minimum bias(MB) events are selected as data sample which was obtained with AND logic of V0A and V0C. 

As event selection for data sample in Pb--Pb collisions, three different trigger selections(kCentral(0-10\%), kSemiCentral(10-50\%) and kMB(0-90\%)) are applied and corresponding number of events are 24.8 M events in central event, 21.8 M events in semi-central events and 3.5 M events in minimum-bias events.

\subsubsection{Topological selection}
\subsubsection{Particle identification}


\subsection{Efficiency correction}
\subsection{Systematic uncertainties}
\subsection{\xis transverse momentum spectra}

