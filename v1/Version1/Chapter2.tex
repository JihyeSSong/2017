\section{Production of resonance with strangeness}
The Quark Model, proposed independently by Murray Gell-Mann and Yuval Ne?eman in 1964 \cite{cite:gellmann}, enables the classification of hadrons in terms of their constituent quarks. In this model, the lighter mesons and baryons are representations of an SU$_{f}$(3) group, whose fundamental representation is the three dimensional vector (u, d, s). These are the three lighter quarks whose characteristics are reported in Table ref{table:quark}. 


\begin{table}[h!]
\centering
\begin{tabular}{lclclc|c|}
\hline
Light flavor &   d  & u  & s \\
\hline \noalign{\smallskip}
Baryon number (B) &  +1/3     & +1/3  &  +1/3\\
Electric charge (Q) &   -1/3     &  +2/3 &   -1/3 \\
Isospin (I)               &   -1/2     &  +1/2 &     0\\
Strangeness (S)     &     0   & 0 & -1\\
mass (\mmass)   &    2.3$_{-0.5}^{+0.7}$    & 4.8$_{-0.3}^{+0.5}$  &  95$\pm$5\\
\hline\noalign{\smallskip}
\noalign{\smallskip}
\end{tabular}
\caption{Quantum numbers and masses associated to the three lighter quarks: u, d and s}\label{table:quark}
\end{table}

The hadronic state are obtained from the decomposition of the following scalar products of the fundamental representations of the group: \\

Meson (q$\bar{q}$) 3 $\bigotimes$ $\bar{3}$ =  1 $\bigoplus$ 8 \\

Baryon (qqq) 3 $\bigotimes$ 3 $\bigotimes$ 3  =  10 $\bigoplus$ 8 $\bigoplus$ 8 $\bigoplus$ 1 \\




\subsection{Strange quark and hyperons}


\subsection{Resonance prodction}